\hypertarget{pipeline-description}{%
\subsection{Pipeline description}\label{pipeline-description}}

BabelBrain takes 3D imaging data (MRI and, if available, CT) along with
a trajectory indicating the location and orientation of an ultrasound
transducer to target a location in the brain.

Currently, four types of transducers are supported:

\begin{itemize}
\tightlist
\item
  \textbf{Single}. This is a simple focusing single-element transducer.
  The user can specify diameter, focal length and a frequency between
  100 kHz and 700 kHz.
\item
  \textbf{H317}. This is a 128-element phased array with a focal length
  of 135 mm and F\#=0.9. The device is capable to operate at 250 kHz and
  700 kHz.
\item
  \textbf{CTX\_500}. This is a device commercialized by the company
  NeuroFUS that has 4 ring elements, with a focal length of 63.2 mm and
  F\# = 0.98, and operates at 500 kHz.
\item
  \textbf{H246}. This is a flat ring-type device that has 2 ring
  elements, with a diameter of 33.6 mm and operates at 500 kHz. It
  offers some degree of focusing by using two transducer elements.
\end{itemize}

The specific capabilities of each transducer are considered during the
simulations.

\hypertarget{preliminary-steps}{%
\section{1 - Preliminary steps}\label{preliminary-steps}}

\begin{itemize}
\item
  \textbf{Mandatory}: Collect T1W and T2W imaging of a participant.
  Highly recommended to use 3D isotropic (1 mm-resolution) scans.
\item
  \textbf{Mandatory}: Execute SimNIBS 3.x \texttt{headreco} or SimNIBS
  4.x \texttt{charm} processing tool:

\begin{verbatim}
headreco all <ID> <Path to T1W Nifti file> <Path to T2W Nifti file>
\end{verbatim}

  or

\begin{verbatim}
charm <ID> <Path to T1W Nifti file> <Path to T2W Nifti file> --forceqform
\end{verbatim}

  \texttt{\textless{}ID\textgreater{}} is a string for identification. A
  subdirectory \texttt{m2m\_\textless{}ID\textgreater{}} will be
  created. Take note of this directory, this will be referred to as the
  \textbf{SimNIBS output} directory in the following of this manual. The
  \texttt{-\/-forceqform} parameter is required as often Nifti files are
  not 100\% strict on how qform and sform matrices are saved.
\item
  \textbf{Mandatory}: Identify the coordinates of the target of focus
  ultrasound in T1W space. If you need to start in standardized space
  (e.g.~MNI), there are many tools (FSL, SPM12, etc) that can be used to
  convert from standardized space to T1W space.

  For example, with FSL, a simple csv file (\texttt{mni.csv}) can be
  created with the coordinates in MNI such as
  \texttt{-32.0\ -20.0\ 65.0}. Then run the following commands

  \texttt{flirt\ -in\ \textless{}path\ to\ T1W\ nifti\textgreater{}\ -ref\ \$FSLDIR/data/standard/MNI152\_T1\_1mm\ -omat\ anat2mni.xfm\ -out\ anat\_norm}

  \texttt{std2imgcoord\ -img\ \textless{}path\ to\ T1W\ nifti\textgreater{}\ \ -std\ \$FSLDIR/data/standard/MNI152\_T1\_1mm.nii\ -xfm\ anat2mni.xfm\ mni.csv\ \textgreater{}\ natspace.csv}

  The file \texttt{natspace.csv} will contain the MNI coordinates
  converted to T1W space. Please note that often visual inspections
  could be required to confirm the location.
\item
  \emph{Optional}: CT scan of the participant. Depending on the study
  being conducted, counting with a CT scan improves the precision of the
  simulation.
\item
  \emph{Optional}:: ZTE scan of the participant. A pseudo-CT scan can be
  reconstructed using a Zero Echo Time (ZTE) MRI scan. Details on MRI
  scan parameters and methods for pseudo-CT reconstruction (using the
  ``classical'' approach) can be found in the work presented by
  \href{https://ieeexplore.ieee.org/document/9856605}{Miscouridou
  \emph{et al.}} (DOI: 10.1109/TUFFC.2022.3198522). The user needs only
  to provide the Nifti file of the ZTE scan. BabelBrain will do the
  transformation to pseudo-CT as detailed in Miscouridou \emph{et al.} A
  Nifti file with the pseudo-CT will be generated.
\end{itemize}

\hypertarget{a---availability-of-ctzte-scans}{%
\subsection{1.a - Availability of CT/ZTE
scans}\label{a---availability-of-ctzte-scans}}

If no CT or ZTE scans are available, a mask representing the skull bone
will be generated from the \texttt{headreco} or \texttt{charm} tools
output. Be sure of inspecting the generated mask Nifti file to ensure
the mask is correctly calculated. Our experience indicates that
\texttt{charm} tool produces a better skull mask extraction. When using
only T1W and T2W as inputs, BabelBrain uses a generic mask to represent
the skull bone (including regions of trabecular and cortical bone).
Average values of speed of sound and attenuation are assigned to these
bone layers. Consult the appendix section for details on the values
used.

If a CT or ZTE scan is provided, a mapping of density, speed of sound
and attenuation will be produced. Consult the appendix section for
details on the mapping procedure.

\hypertarget{planning}{%
\section{2 - Planning}\label{planning}}

The goal of the planning step is to produce a \textbf{trajectory} that
provides the location where ultrasound is intended to be focused and the
orientation of the transducer in T1W coordinate space. In practice, the
trajectory is just an affine matrix applied to a ``virtual'' needle that
describes the \textbf{location} and \textbf{orientation} where focused
ultrasound is desired to be concentrated. The tip of the trajectory
needs to be at the intended target. The position of the transducer will
be relative to the tip location. The details using 3DSlicer can
illustrate this.

\hypertarget{acoustic-path-stl-helpers}{%
\paragraph{Acoustic path STL helpers}\label{acoustic-path-stl-helpers}}

BabelBrain includes a series of complementary STL files representing the
``acoustic'' path. Each STL file includes a group of circular meshes
combined with a target needle that represent the acoustic cross sections
of a field produced with a transducer with F\#=1.0 at different depths.
As noted in the instructions below, these meshes help to verify a
correct alignment with the skin.

\hypertarget{a---planning-with-3dslicer}{%
\subsection{2.a - Planning with
3DSlicer}\label{a---planning-with-3dslicer}}

\begin{enumerate}
\def\labelenumi{\arabic{enumi}.}
\item
  Install the \textbf{SlicerIGT} extension in 3DSlicer (restart 3DSlicer
  if requested)
\item
  Load T1W planning data
\item
  In the IGT extension menu, select ``Create Models''
\item
  Load one of the STL helpers as a \texttt{model} with \texttt{RAS}
  coordinate convention. The model will appear by default centred in the
  T1W space and pointing in the inferior\(\rightarrow\)superior
  direction

  Alternatively, you can create a needle with a length of 100 mm.
\item
  Select the model in the data panel and edit the properties to make it
  appear in the ``Slice Display''
\item
  Create a new transform and give it a name related to the target
  (e.g.~LGPI, RSTN, LVIM, RM1, etc.). This is important as BabelBrain
  will use the name of the transform as a prefix for its output files.

  Apply the transform to the model and be sure the transformation is set
  to \texttt{local} (little button next to the ``invert'' button)
\item
  Select ``Volume Reslice Driver'' in the IGT module menu
\item
  Select the linear transform in the two first slice views
\item
  Select one view to be ``Inplane'' and the other to be ``Inplane 90''
\item
  In the Data panel, select the linear transform and edit properties,
  you should be able to see the slice views aligned along the model
\item
  Adjust the location of the tip of the needle using the
  \textbf{translation} (LR, PA, IS) controls to match the tip of the
  model to your area of interest.
\item
  Adjust the trajectory path using the \textbf{rotation} (LR, PA, IS)
  controls until finding a trajectory that has a clear path and mimics
  how the transducer will be placed. Tip: Adjust the trajectory to make
  it orthogonal to the skin surface in the inline and inline90 views;
  this recreates the condition of placing a transducer aligned relative
  to the skin.

  Note: If you navigate to other windows in 3DSlicer, the transition and
  rotation control may be set back to 0s. But the transformation matrix
  will remain with the latest values applied. Any other adjustment will
  be added to the transformation matrix. Be sure that the \texttt{local}
  option is always selected.
\item
  Save the transformation in text format. Select ``Save data'' and
  select text as the file format. Take note of the path. Suggestion:
  Select a directory in the same path where T1W or SimNIBS output is
  located.
\end{enumerate}

\hypertarget{b---planning-with-brainsight}{%
\subsection{2.b - Planning with
Brainsight}\label{b---planning-with-brainsight}}

Alternatively, planning can also be performed with the proprietary
software Brainsight made by Rogue Research (Montreal, Canada) for the
planning and execution of non-invasive neuromodulation. This software
has an existing feature that exports a trajectory that can be used in
BabelBrain. The workflow to export a trajectory is very similar to
3DSlicer.

\begin{enumerate}
\def\labelenumi{\arabic{enumi}.}
\tightlist
\item
  Create a new ``empty'' or ``SimNIBS'' project; use SimNIBS only if you
  used SimNIBS 3.x with \texttt{headreco}.
\end{enumerate}

\begin{enumerate}
\def\labelenumi{\arabic{enumi}.}
\setcounter{enumi}{1}
\item
  Load T1W planning data. If using ``SimNIBS'' project, it will preload
  the T1W imaging dataset.
\item
  Open target window
\item
  Adjust coordinates and orientation with control in the user interface
  (right side of screen)
\item
  Create a new target as a trajectory
\item
  Rename the trajectory with a name related to the target (e.g.~LGPI,
  RSTN, LVIM, RM1, etc.)
\item
  Export trajectory with ``Export'' function and select ``Orientation (3
  directions vectors)'' and ``NifTI:Scanner'' as the coordinate system.
  Take note of the path. Suggestion: Select a directory in the same path
  where T1W or SimNIBS output is located.
\end{enumerate}

\hypertarget{simulation-with-babelbrain}{%
\section{3 - Simulation with
BabelBrain}\label{simulation-with-babelbrain}}

Now that planning is done, open BabelBrain either from the Applications
menu in macOS if the DMG installer was used or with
\texttt{python\ BabelBrain.py} as indicated in the installation section.

\hypertarget{a---input-data}{%
\subsection{3.a - Input data}\label{a---input-data}}

An input dialog will prompt the different input files required for the
simulation.

\begin{enumerate}
\def\labelenumi{\arabic{enumi}.}
\item
  Specify the path to the trajectory file and the source (Slicer or
  Brainsight)
\item
  Select the SimNIBS output directory associated to this test and
  indicate what tool was used to generate it (\texttt{headreco} or
  \texttt{charm})
\item
  Select the path to the T1W Nifti file
\item
  Indicate if CT scan is available. Options are ``No'', ``real CT'' or
  ``ZTE''. Select if coregistration of CT to T1W space must be
  performed. Depending on your specific preliminary steps, you may have
  CT already coregistered in T1W space. If coregistration is done by
  BabelBrain, the resolution of the CT will be preserved. The T1W file
  will be first bias-corrected and upscaled to the CT resolution and
  then the CT will be coregistered using the \texttt{itk-elastix}
  package with rigid coregistration.
\item
  Select a thermal profile file for simulation. This is a simple YAML
  file where the timings of transcranial ultrasound are specified. For
  example:

\begin{verbatim}
BaseIsppa: 5.0 # W/cm2
AllDC_PRF_Duration: #All combinations of timing that will be considered
-   DC: 0.3
    PRF: 10.0
    Duration: 40.0
    DurationOff: 40.0
\end{verbatim}

  This definition helps in the step of thermal simulation with
  BabelBrain. \texttt{BaseIsspa} is the reference value of acoustic
  intensity for which the thermal equation will be solved. You can set
  this to 5 W/cm\(^2\). Choices for other powers will be scaled (no
  recalculations) based on this value.

  More than one exposure can be specified. For example:

\begin{verbatim}
BaseIsppa: 5.0 # W/cm2
AllDC_PRF_Duration: #All combinations of timing that will be considered
    -   DC: 0.3
        PRF: 10.0
        Duration: 40.0
        DurationOff: 40.0
    -   DC: 0.1
        PRF: 5.0
        Duration: 80.0
        DurationOff: 50.0
\end{verbatim}

  When running the thermal simulation step, all the combinations
  specified in the thermal profile will be calculated.
\item
  Select the type of transducer to be used in simulations.
\item
  Once all inputs are set, then click on ``CONTINUE'' \#\# 3.b - Domain
  generation The diagram below shows flowchart describing the process
  for the domain generation.
\end{enumerate}

The first step after specifying input data is to create the simulation
domain. The available operating frequencies will depend on the selected
transducer. The second main input is the resolution of the simulation
expressed in the number of points per wavelength (PPW). The minimum for
fast estimation is 6 PPW, and 9 PPW to meet criteria de convergence when
compared to other
\href{https://asa.scitation.org/doi/10.1121/10.0013426}{numerical
tools}. Depending on if CT or ZTE scans are available, options to
fine-tune the domain generation will be available. For CT scans, the
user can adjust the threshold for bone detection (set by default to 300
HU). For ZTE scans the user can specify the thresholds to select
normalized ZTE signal (by default 0.1 and 0.6) to convert to pseudo-CT.
Please consult Miscouridou \emph{et
al.}{]}(https://ieeexplore.ieee.org/document/9856605) for details on the
``classical'' approach to convert from ZTE to pseudo-CT. The execution
time in M1 Max processor can take from 1 minute of minutes up to 10
minutes depending on the resolution and availability of ZTE/CT scans.
When initiating the calculations, a detailed log output will appear in
the bottom region of the window. In case of any error during processing,
a dialog message will prompt indicating to consult this window for more
details. Once executed, orthogonal views of the domain will be shown.
T1W scan is also shown to verify that the mask was correctly calculated.

Once executed, a Nifti file containing the mask describing the different
tissue regions will be produced in the directory where the T1W Nifit
file is located. It will have a file with the following structure:
\texttt{\textless{}Name\ of\ target\ file\textgreater{}\_\textless{}Frequency\textgreater{}\_\textless{}PPW\textgreater{}\_BabelViscoInput.nii.gz},
for example
\texttt{LinearTransform\_500kHz\_6PPW\_BabelViscoInput.nii.gz}. The mask
will be in T1W space, facilitating its inspection as overlay with T1W
data. The mask has values of 1 for skin, 2 for cortical bone, 3 for
trabecular and 4 for brain tissue. A single voxel with a value of 5
indicates the location of the target. The raw data inside the Nifti file
is organized in a 3D Cartesian volume that is aligned to the transducer
acoustic axis. The Nifti affine matrix ensures the mask can be inspected
in T1W space.

If a CT or ZTE dataset is indicated as input, the skull mask will be
created using this scan rather than the output of \texttt{headreco} or
\texttt{charm}. Also, an overlay of the CT/pseudo-CT will be also shown
for verification purposes.

Please note if a
\texttt{\textless{}Name\ of\ target\ file\textgreater{}\_\textless{}Frequency\textgreater{}\_\textless{}PPW\textgreater{}\_BabelViscoInput.nii.gz}
file exists, the GUI will ask confirmation to recalculate the mask.
Selecting ``No'' will load the previous mask.

If using output from \texttt{headreco}, the STL files for skin, csf and
bone are used to produce the high-resolution mask via GPU-accelerated
voxelization.

If using output from \texttt{charm} (which does not produces STL files),
equivalent STL files are produced from the file
\texttt{final\_tissues.nii.gz} created by charm. Meshes are created and
smoothed (Laplace filtering), and the mask for simulation is calculated
via GPU-accelerated voxelization. The STL files of skin, csf and bone
will be saved in the output directory of SimNIBS by BabelBrain.

\hypertarget{c---transcranial-ultrasound-simulation}{%
\subsection{3.c - Transcranial ultrasound
simulation}\label{c---transcranial-ultrasound-simulation}}

The second tab in the GUI of BabelBrain shows the ultrasound simulation
step. The diagram below shows a flowchart of this step.

The choices of this tab will depend on the selected transducer.
Simulation results in this step are shown in normalized conditions. The
final step (see below) later will show the results denormalized in
function of the selected intensity at the target. Common to all
transducers, the distance of the maximal depth beyond the target
location is set to a user-configurable distance of 40 mm.

\hypertarget{c.i---ctx_500}{%
\subsubsection{3.c.i - CTX\_500}\label{c.i---ctx_500}}

For the CTX\_500 transducer, the initial assumption is that this type of
transducer will be placed in direct contact with the skin and that the
focusing distance will be adjusted according to the desired target.

The initial ``TPO Distance'' (an adjustable parameter in the CTX\_500
device) is calculated based on the distance skin to the target.

It is recommended to simulate with the default values to evaluate the
degree of focus shift caused by the skull. Simulation should take a
couple of minutes in a M1 Max system.

The results window will show two orthogonal views of normalized acoustic
intensity. The intensity in the skin and skull regions is masked out (it
can be visualized later in those regions in step 3). In this step, the
main goal is to ensure a correct spatial focusing on the target. In the
example, a shift of 5 mm of the focal spot towards the transducer can be
observed. This shift can be corrected by adding 5 mm in the TPO Distance
input (in the example, we adjust to 52.5 mm). Also, there is a small
lateral shift in the negative ``Y'' direction. This can be corrected
with the ``Mechanical'' adjustment controls (in this example we adjust
+1mm in the Y direction). Please note that in the simulation domain, X,
Y and Z are not mapped to subject coordinates. However, at the end of
the simulations, there will be a report in which direction in the T1W
space this adjustment translates.

After doing the adjustments, the simulation can be repeated.

\hypertarget{c.ii---h246}{%
\subsubsection{3.c.ii - H246}\label{c.ii---h246}}

The H246 transducer has a similar operation as the CTX\_500. The steps
presented above apply similarly. As the H246 transducer has a much
longer focal length, consider extending the maximal depth of
simulations.

\hypertarget{c.iii---h317}{%
\subsubsection{3.c.iii - H317}\label{c.iii---h317}}

The H317 is a large transducer that uses a coupling cone that is in
contact with the skin. The user interface shows small differences
compared to CTX\_500 and H246. There is a parameter for the
\texttt{Distance\ cone\ to\ Focus} that depends on the acoustic cone
used for coupling. Because this transducer has 128 elements, the user
interface shows also the option to perform electronic refocusing.

\hypertarget{c.iv---single}{%
\subsubsection{3.c.iv - Single}\label{c.iv---single}}

The ``Single'' transducer is a generic device with a configurable
diameter and focal length. Because this is a more general-purpose
device, it is not assumed that the transducer is in direct contact with
the skin. The transducer is always initially centered at the target,
which can make that there could be some space between the transducer out
plane and the skin. The user can adjust the mechanical distance on the Z
axis until the point the out plane of the transducer reaches the skin.

\hypertarget{d---thermal-simulation}{%
\subsection{3.d - Thermal simulation}\label{d---thermal-simulation}}

The third tab in the GUI of BabelBrain shows the thermal simulation
step. The diagram below shows a flowchart of this step.

The thermal simulation solves the Bio-heat thermal equation (BHTE) for
all the combinations of duty cycle, timing and ultrasound exposure
indicated in the thermal profile definition file.

The selection of spatial-peak pulse-average intensity
(\(I_{\text{SPPA}}\)) indicates the desired intensity at the target. The
spatial-peak time-average intensity (\(I_{\text{SPTA}}\)) is calculated
based on the selected timing conditions. Based on the selections of
timing and desired \(I_{\text{SPPA}}\) in tissue, the
\(I_{\text{SPPA}}\) in water conditions is calculated after taking into
account all the losses. Thermal safety parameters (maximal temperature
and thermal doses) in the skin, skull bone and brain tissue are
calculated at the locations showing the highest temperature elevation in
the whole 3D volume. The \texttt{MTB}, \texttt{MTS} and \texttt{MTC}
push buttons in the lower region of the interface select the slice
corresponding to the the maximal temperature in brain, skin and skull,
respectively.

The \texttt{Export\ summary\ (CSV)} action exports the input data paths
and user selections used for the simulations. It also includes a table
of \(I_{\text{SPPA}}\) in water conditions and safety metrics in
function of the desired \(I_{\text{SPPA}}\) in tissue. Below there is an
example of the exported data.

\begin{longtable}[]{@{}
  >{\raggedright\arraybackslash}p{(\columnwidth - 18\tabcolsep) * \real{0.0787}}
  >{\raggedright\arraybackslash}p{(\columnwidth - 18\tabcolsep) * \real{0.1348}}
  >{\raggedright\arraybackslash}p{(\columnwidth - 18\tabcolsep) * \real{0.0674}}
  >{\raggedright\arraybackslash}p{(\columnwidth - 18\tabcolsep) * \real{0.0787}}
  >{\raggedright\arraybackslash}p{(\columnwidth - 18\tabcolsep) * \real{0.0674}}
  >{\raggedright\arraybackslash}p{(\columnwidth - 18\tabcolsep) * \real{0.0674}}
  >{\raggedright\arraybackslash}p{(\columnwidth - 18\tabcolsep) * \real{0.0674}}
  >{\raggedright\arraybackslash}p{(\columnwidth - 18\tabcolsep) * \real{0.1461}}
  >{\raggedright\arraybackslash}p{(\columnwidth - 18\tabcolsep) * \real{0.1461}}
  >{\raggedright\arraybackslash}p{(\columnwidth - 18\tabcolsep) * \real{0.1461}}@{}}
\toprule()
\begin{minipage}[b]{\linewidth}\raggedright
Isppa
\end{minipage} & \begin{minipage}[b]{\linewidth}\raggedright
IsppaWater
\end{minipage} & \begin{minipage}[b]{\linewidth}\raggedright
MI
\end{minipage} & \begin{minipage}[b]{\linewidth}\raggedright
Ispta
\end{minipage} & \begin{minipage}[b]{\linewidth}\raggedright
MTB
\end{minipage} & \begin{minipage}[b]{\linewidth}\raggedright
MTS
\end{minipage} & \begin{minipage}[b]{\linewidth}\raggedright
MTC
\end{minipage} & \begin{minipage}[b]{\linewidth}\raggedright
CEMBrain
\end{minipage} & \begin{minipage}[b]{\linewidth}\raggedright
CEMSkin
\end{minipage} & \begin{minipage}[b]{\linewidth}\raggedright
CEMSkull
\end{minipage} \\
\midrule()
\endhead
0.5 & 2.38 & 0.18 & 0.15 & 0.03 & 0.01 & 0.02 & 0.00033419 & 0.000329626
& 0.00033046 \\
1 & 4.77 & 0.25 & 0.3 & 0.07 & 0.03 & 0.04 & 0.000343142 & 0.000333787 &
0.000335482 \\
1.5 & 7.15 & 0.31 & 0.45 & 0.10 & 0.04 & 0.05 & 0.000352388 &
0.000338007 & 0.000340589 \\
2 & 9.54 & 0.35 & 0.6 & 0.13 & 0.06 & 0.07 & 0.00036194 & 0.000342286 &
0.000345782 \\
2.5 & 11.92 & 0.39 & 0.75 & 0.17 & 0.07 & 0.09 & 0.000371807 &
0.000346625 & 0.000351062 \\
3 & 14.31 & 0.43 & 0.9 & 0.20 & 0.09 & 0.11 & 0.000382002 & 0.000351024
& 0.000356431 \\
\ldots{} & \ldots{} & \ldots{} & \ldots{} & \ldots{} & \ldots{} &
\ldots{} & \ldots{} & \ldots{} & \ldots{} \\
\bottomrule()
\end{longtable}
